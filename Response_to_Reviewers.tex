%% start of file `template.tex'.
%% Copyright 2006-2013 Xavier Danaux (xdanaux@gmail.com).
%
% This work may be distributed and/or modified under the
% conditions of the LaTeX Project Public License version 1.3c,
% available at http://www.latex-project.org/lppl/.


\documentclass[11pt,a4paper,sans]{moderncv}        % possible options include font size ('10pt', '11pt' and '12pt'), paper size ('a4paper', 'letterpaper', 'a5paper', 'legalpaper', 'executivepaper' and 'landscape') and font family ('sans' and 'roman')

\usepackage[document]{ragged2e}
% pour justifier


% moderncv themes
\moderncvstyle{banking}                            % style options are 'casual' (default), 'classic', 'oldstyle' and 'banking'
\moderncvcolor{red}                                % color options 'blue' (default), 'orange', 'green', 'red', 'purple', 'grey' and 'black'
\renewcommand{\familydefault}{\rmdefault} 
%\nopagenumbers{} 
\usepackage[utf8]{inputenc} 
\usepackage[scale=0.9]{geometry}


\def \draft {1}
\usepackage{xparse}
\usepackage{ifthen}
\DeclareDocumentCommand{\comment}{o m o o o o}
{\ifthenelse{\draft=1}{
  \IfValueT{#1}{
      \textcolor{red}{\textbf{C (#1) : }#2}
      \IfValueT{#3}{\textcolor{blue}{\textbf{A1 : }#3}}
      \IfValueT{#4}{\textcolor{green}{\textbf{A2 : }#4}}
      \IfValueT{#5}{\textcolor{red!50!blue}{\textbf{A3 : }#5}}
      \IfValueT{#6}{\textcolor{blue}{\textbf{A4 : }#6}}
    }
    \IfNoValueT{#1}{
      \textcolor{red}{\textbf{C : }#2}
      \IfValueT{#3}{\textcolor{blue}{\textbf{A1 : }#3}}
      \IfValueT{#4}{\textcolor{green}{\textbf{A2 : }#4}}
      \IfValueT{#5}{\textcolor{red!50!blue}{\textbf{A3 : }#5}}
      \IfValueT{#6}{\textcolor{blue}{\textbf{A4 : }#6}}
    }
 }{}
}


\firstname{}
\lastname{}
\begin{document}

% recipient data
\recipient{Editor JASSS}{}
\date{\today}
\opening{Dear Editor,}
\closing{Yours faithfully,\\
Juste Raimbault and co-authors%\\
%Université Paris 7 - UMR CNRS 8504 Géographie-cités
}
         % use an optional argument to use a string other than "Enclosure", or redefine \enclname

%\makelettertitle



\justify



\textbf{Response to reviewers}

%Resubmit 15 Aug. 2019\\

\bigskip

We want to thank the two reviewers for their thorough evaluation and discussion of our paper, as well as the challenging questions they asked. We have revised the manuscript in a way that we believe addresses most of the comments. In this response, we first want to address the two main questions that arose for both reviews. The first one (A) relates to the original mis-specification/ambiguities between the social and spatial components of the modelling process. The second one (B) relates to the contribution of the approach in terms of potential use for model makers, for domain experts and in operational contexts.

 \medskip
A. In terms of the ambiguity between the social and spatial component 
\begin{itemize}
\item A.1. We have precised and clarified the argument of the paper (mainly in introduction) by qualifying our contribution, which is to provide an operational framework for studying the influence of geometric structures of space on the results of a geosimulation model. We have thus distinguished two effects of geographical space in simulation modelling: the ones that are part of the model, reflecting for example the role played by distance on decreasing agents' perception or potential social interactions; and the ones which act as geometric structures "on which" the social elements simulated interact, i.e. geometric constraints of the simulation. The first type of effects is addressed by traditional modelling. Our original contribution is to tackle the second type.
 
\item A.2. We have clarified the terms used in our paper introduction to distinguish between: \textbf{sociospatial systems} (i.e. groups of social agents whose behaviour is constrained by their position in geographical space), \textbf{interactions} (what happens between agents and between agents and their environment, in the model as well as in the empirical system) and \textbf{geometric structure}, a term we use for qualifying the geometric outputs of the spatial generator which are used as initial spatial conditions in the model. When we talk about the generator we actually used in the paper, we call it a "density grid generator" because it generates a square grid with heterogeneous density, whereas in all generality a "spatial generator" could potentially generate graph or point structures.

 \item A.3. We have modified figure 1 to express the clarified approach visually. Naming the generator parameter "gamma" has helped reduced the confusion that the term 'meta-parameter' was inducing. We have also replaced the expressions "quantitative results" by "sensitivity analysis" and "qualitative results" by "variations by types of geometric structure", in the figure as in the subsections of the paper.
\end{itemize}

 \medskip
B. To highlight limitations as well as promising perspectives of improvement, we answer the concern about the genericity of the method and expose its links with the conceptual and empirical domains:

 \begin{itemize}
 
  \item B.1. We have added a discussion on how segregation indicators can themselves be linked with geometric structure (which can typically happens if an indicator give the same weight to cells of different densities). This is a limitation of the method as it is but also an avenue for promoting stronger indicators in empirical research. 

  \item B.2. To inform the interactions between empirics, concepts and models as discussed by Livet et al., 2010, we added a paragraph in discussion, explaining how our approach allows us to go back to questioning empirical cases from the model and questioning our concepts (for example how 'neighbourhood' is modelled since Schelling compared to the meaning of neighbourhood in empirical/theoretical social sciences).
  
  \item B.3. As a perspective to reduce this bias, we could compute ego-neighbourhoods rather than a regular grid, and make these systematically vary and study the effect on emergent behavior and indicators. In further researches, we plan to extend the analysis to vary the size of grid and its cell shape (cf. MAUP), but also generate other types of geometric structures to feed into geosimulation models (for example points or networks).
\end{itemize}

\bigskip


\medskip
\textbf{Point-by-point response to referees comments:}

\medskip

\textbf{First referee:}

\medskip

%The paper describes an approach that takes spatial sensitivity analysis explicitly and thoroughly into account. Based on the criticism that most agent-based-modelling and simulation approaches do neglect a careful recognition of the spatial impact on social phenomena, it introduces a “spatial generator” whose task is to generate different possible spatial configurations that are then used as a comparative measurement in socio-spatial simulations to prove the influence of spatial structures. To evaluate the strengths and weaknesses of the use of the spatial generator tool the author(s) applies two classical approaches – Schelling’s segregation model and the Sugarscape model – as benchmark models. 

\begin{enumerate}
 % The methodology described is very convincing as it allows the utilisation of a tool that takes the spatial conditions – as causes or effects or intermediates – actually explicitly into account. With this approach, the paper convincingly outlines a pragmatic application by creating typical spatial compositions in urban contexts that is monocentric, polycentric or discontinuous cities.
 	\item \textit{From this methodological perspective of reviewing, it would be helpful to add some more information to approaches/techniques used within the tool of the spatial generator. For example, paragraph 2.8 mentions four dimensions of spatial structure but leave the reader alone with just the terms. Either describe these or omit it.}
   
   $\rightarrow$ To describe in more details the density grid generator we added a few sentences; and the generator was furthermore better situated among the many possible others in the discussion.
   %ways of making a spatial generator as Young2002 (FLN/MLT).
 	% item \comment[JR]{will add some description on urban form indicators etc.}
	  
  \medskip
  
   \item \textit{This also applies to the three segregation indexes: what exactly do they measure?}
  
  $\rightarrow$ Segregation indexes were better described in the section describing the models.
  
  \medskip
  
  \item \textit{The description of the spatial generator’s procedure, outlined in paragraph 2.7, would also benefit if more information is added} \ldots
  %, for example, to delineate the terms “strength of attraction” (attraction in which respect?) and “strength of the diffusion process” (is strength equal to intensity? If so, is it then in space and/or time?). 
  
  $\rightarrow$ The description of the spatial generator was partly reformulated with more details.

  \medskip

  \item \textit{While the methodological part of the paper is convincing the theoretical and epistemological background is less so. In short, the paper would improve significantly, if most of the geographical and empirical associations would be left out.}
  %Even claimed by the author(s), the approach has not much to do with the empirical growth or development of cities, just because the approach reduces cities’ growth to the growth itself, omitting issues such as historical genesis, cultural differentiation, economic background political and legal circumstances, etc. 
   
   $\rightarrow$ We have clarified the semantic terms used in sections 1.5 to 1.8 and in the figure 1.
   
 

\medskip

  \item \textit{The problem, from a geographical point of view, is that the paper switches between “geographical” and “spatial” without making any differentiation. In so doing, the author(s) runs immediately into the problem of adequate definitions. In paragraph 1.2 the definition of geographical systems as “social agents interacting with one another, within a limited portion of space” is misleading. First, it is a tautology (geographical systems in space); secondly, geographical cannot be described with social agents, because geographical systems are composed of spatial units, their relations, and a spatially defined system border. Social systems, in turn, might be described with humans as elements, their relations, and again a socially defined system border. Moreover, with respect to sociological system theory (Luhmann 1993), space does not add any knowledge to the nature of social systems. This approach can be (and I would add: has to be) criticised, however, not by interming! ling fuzzily the social with the geographical, as happened with this definition.} 
 
 $\rightarrow$ This issue is tackled by comment A.
 
 \medskip

  

  \item \textit{Then, in paragraph 1.3, the complexity of systems with respect to segregation is also problematically described.} \ldots
  
  $\rightarrow$ see comment A. As for the role of geometric structures in the production of segregation, Banos (2012) has shown that with the same Schelling mechanisms and the same parameter values but different network structures of neighbourhoods (scale-free, regular, etc.) the segregation observed at the end of simulation varied significantly between network structures. This is the kind of approach we systematise here, although with a spatial grid instead of network structures.

  \medskip

  \item \textit{The problem of setting space and geography as synonymous continues in paragraphs 1.9 and following. What is described here as "spatial system shape" is the geometry of spaces but not the geography of spaces.}

  $\rightarrow$ We agree and corrected the semantics (see comment A). 

  \medskip

  \item \textit{Even more critically is the arbitrary conflation of social and spatial units as happened in paragraph 2.1, as geographical entities are described with “people, housing, networks, etc.” One is inevitably asking oneself: do geographical entities comprise of simply everything? Obviously not, but the paper seems to argue this way.}
  
  $\rightarrow$ All these terms were clarified (see comment A). 

  \medskip

  \item \textit{Due to this, attempts to explain the growth of existing cities should be avoided.}
  
  $\rightarrow$ We agree with this comment and recall that urban growth processes are not the main focus of the paper, they are here an instantiation of a given $\gamma$ generator, which could have been replaced by others such as network models, procedural modeling, etc. The position and discussion of the paper was amended to precise that (see comment A).

 
  \medskip
 
 
  \item \textit{The approach is reductionist and has its strength in reflecting growth at an abstract level nevertheless. The concluding remark in paragraph 5.1 ``With the Schelling experiment, we found \ldots that polycentric and discontinuous cities appear systematically more segregated than compact cities\ldots'' is elusive. Either it is a trivial fact, simply because polycentric cities have more centres and thus, by nature, more opportunities from which segregation can originate. Or, the results can be used as a starting point to dig deeper in geographical reasons, taking social, economic, cultural, and political, but also topographical issues into consideration. This geographical reasoning, however, is outside of the scope of this paper. }

  $\rightarrow$ We understand this statement, however we wish to underline that our objective in the discussion of the Schelling results was not to establish a general result on the links between polycentricity and segregation, which are indeed very complex and multidimensional as advocated by the reviewer. Instead, we aimed at showing what kind of results could be obtained using the framework we suggest. If we do not prove that in general, and in empirical context, polycentricity creates more segregation, we obtained, in the modelling context, under Schelling assumptions, that polycentric geometric structures are more often that not associated with segregated patterns, a result which we do not believe to be solely resulting from the fact that more centers leads to more opportunities of segregation (which is true), because it seems also true that monocentric structures, with higher densities, can more easily accommodate heterogeneity of agents within cells, and hence a more patchy type of segregation which can for instance being observed in London. We included a comment in the discussion section, saying that we hope we have convinced people to add spatial sensitivity analysis during the exploration of their model and that specialist of the thematics would see the opportunity to dig deeper into our results.
  
	\medskip

   \item \textit{Formal issues that have to be solved: the numbering of paragraphs is not coherent. The reference of Wheeler (2006) is missing the source of the article.}
   
   $\rightarrow$ These issues were corrected.
 

\end{enumerate}


\bigskip
\bigskip

\textbf{Second referee:}
 

%This manuscript presents a methodology to explore the effects of spatial parameters in ABM. They basically explore what they called “generator parameters”, to distinguish them from other statistical ones, which include density spatial distribution and patterns. To do so, they compared phase diagrams and calculated the mean of final values. 
%As case-studies, they used the actual distributions of three European cities as a means to constraint the density generation in two archetypical model (Schelling's segregation model and Epstein and Axtell's Sugarscape). 

\medskip


\begin{enumerate}

  \item \textit{While the different outcomes of the two implementation exercises are not the main point, I would suggest authors to provide some more elaborated explanations. The two models have different behavioural assumptions and represent a different 'interaction' context. How can be extrapolate and isolate spatial effects when models have different logics and theoretical purposes?}
  
  $\rightarrow$ See above comment A, discussion has been amended and details added.
  \medskip

  \item \textit{In 4.1, the authors suggest potential developments and point to certain limitations. While limitations are well-discussed, readers who are not specialist of spatial models could benefit from a more comprehensive discussion on potential applications and extensions.}
  
  $\rightarrow$ See comment B above.
  
  \medskip
  
  \textit{For instance, it would be useful to have suggestions on how to use this methodology to run empirical models. Suppose to have a nice dataset and a case of a urban context to be modelled. Could the exploration of different spatial configurations even be used to run 'counter-factual' or 'what if' tests on actual spatial configurations?}
  
  $\rightarrow$ This point was not explicit enough and we revised the presentation of our approach (which is generic and not empirical) in introduction (especially fig. 1), as already stated in comment B.3 above. We thank reviewer 2 for highlighting this limitation. 

  \medskip

  \item \textit{Furthermore, there are now examples of spatial models that go beyond administrative neighborhoods and overimposed spatial boundaries, i.e., egohoods (e.g., https://onlinelibrary.wiley.com/doi/abs/10.1111/1745-9125.12006), which could be interesting to see implemented in ABM.}
  
  $\rightarrow$ We added a paragraph on the limits of our density grid approach in discussion, in which we mention the possibility to use egohoods to better measure segregation. Indeed, at present, the use of a grid of varying density means that in dense areas, agents have many neighbours and thus a higher probability of mixing, which affects segregation measurement. With egohoods, each agents would have the same number of neighbours taken into account for the calcultation, but their spatial scope would vary (for some agents, it would be for example 100 neighbours from the same cell whereas for others, it would be the nearest 100 neighbours taken from 25 surrounding cells).
  
  \medskip
  
   \item \textit{Spatial econometrics based on empirical dataset is developing fast. It would be useful to understand how these models can enrich ABM and vice-versa.}
   
     $\rightarrow$ This points suggest that our work can inform the interactions between empirics, concepts and models as discussed by Livet et al., 2010. We added a paragraph on this in the discussion, and suggested how our approach allows to go back to questioning empirical cases from the model and questioning our concepts ('neighbourhood').

  \medskip

  \item\textit{Furthermore, even among limitations, perhaps authors could discuss data limitations and the fact that social processes (networks, mobility) cannot conflate with spatial dynamics. Here, it is a bit misleading to call "generator parameters" ONLY the spatial ones!}
  
  $\rightarrow$ We have clarified this issue in introduction but referring to different terms and figure 1 which differentiates between the $\mu$ model and $\gamma$ parameters (see comment A).
 
 \medskip
 
 
  \item\textit{In this respect, I would encourage authors to further develop their closing section, which now does not stimulate readers to imagine implications of authors' findings and message to their own research.} \ldots
  
  $\rightarrow$ The discussion and conclusion was amended in this direction (see comment B).

\end{enumerate}



\end{document}


%% end of file `template.tex'.