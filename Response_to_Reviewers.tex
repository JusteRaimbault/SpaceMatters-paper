%% start of file `template.tex'.
%% Copyright 2006-2013 Xavier Danaux (xdanaux@gmail.com).
%
% This work may be distributed and/or modified under the
% conditions of the LaTeX Project Public License version 1.3c,
% available at http://www.latex-project.org/lppl/.


\documentclass[11pt,a4paper,sans]{moderncv}        % possible options include font size ('10pt', '11pt' and '12pt'), paper size ('a4paper', 'letterpaper', 'a5paper', 'legalpaper', 'executivepaper' and 'landscape') and font family ('sans' and 'roman')

\usepackage[document]{ragged2e}
% pour justifier


% moderncv themes
\moderncvstyle{banking}                            % style options are 'casual' (default), 'classic', 'oldstyle' and 'banking'
\moderncvcolor{red}                                % color options 'blue' (default), 'orange', 'green', 'red', 'purple', 'grey' and 'black'
\renewcommand{\familydefault}{\rmdefault} 
%\nopagenumbers{} 
\usepackage[utf8]{inputenc} 
\usepackage[scale=0.9]{geometry}


\def \draft {1}
\usepackage{xparse}
\usepackage{ifthen}
\DeclareDocumentCommand{\comment}{o m o o o o}
{\ifthenelse{\draft=1}{
  \IfValueT{#1}{
      \textcolor{red}{\textbf{C (#1) : }#2}
      \IfValueT{#3}{\textcolor{blue}{\textbf{A1 : }#3}}
      \IfValueT{#4}{\textcolor{green}{\textbf{A2 : }#4}}
      \IfValueT{#5}{\textcolor{red!50!blue}{\textbf{A3 : }#5}}
      \IfValueT{#6}{\textcolor{blue}{\textbf{A4 : }#6}}
    }
    \IfNoValueT{#1}{
      \textcolor{red}{\textbf{C : }#2}
      \IfValueT{#3}{\textcolor{blue}{\textbf{A1 : }#3}}
      \IfValueT{#4}{\textcolor{green}{\textbf{A2 : }#4}}
      \IfValueT{#5}{\textcolor{red!50!blue}{\textbf{A3 : }#5}}
      \IfValueT{#6}{\textcolor{blue}{\textbf{A4 : }#6}}
    }
 }{}
}


\firstname{}
\lastname{}
\begin{document}

% recipient data
\recipient{Editor JASSS}{}
\date{\today}
\opening{Dear Editor,}
\closing{Yours faithfully,\\
Juste Raimbault and co-authors%\\
%Université Paris 7 - UMR CNRS 8504 Géographie-cités
}
         % use an optional argument to use a string other than "Enclosure", or redefine \enclname

%\makelettertitle


%Dear Editor,

\justify


%Thank you for the feedback on the draft paper ``Space Matters: extending sensitivity analysis to initial spatial conditions in geosimulation models''. The suggestions and comments will undoubtedly be of great value to the paper. It was updated accordingly.

\medskip






\textbf{Response to reviewers}


\bigskip


%We now turn to point-by-point responses to referees comments.

\textbf{Point-by-point response to referees comments:}

\medskip

\textbf{First referee:}

\medskip

%The paper describes an approach that takes spatial sensitivity analysis explicitly and thoroughly into account. Based on the criticism that most agent-based-modelling and simulation approaches do neglect a careful recognition of the spatial impact on social phenomena, it introduces a “spatial generator” whose task is to generate different possible spatial configurations that are then used as a comparative measurement in socio-spatial simulations to prove the influence of spatial structures. To evaluate the strengths and weaknesses of the use of the spatial generator tool the author(s) applies two classical approaches – Schelling’s segregation model and the Sugarscape model – as benchmark models. 

\begin{enumerate}
 % The methodology described is very convincing as it allows the utilisation of a tool that takes the spatial conditions – as causes or effects or intermediates – actually explicitly into account. With this approach, the paper convincingly outlines a pragmatic application by creating typical spatial compositions in urban contexts that is monocentric, polycentric or discontinuous cities. From this methodological perspective of reviewing, it would be helpful to add some more information to approaches/techniques used within the tool of the spatial generator. For example, paragraph 2.8 mentions four dimensions of spatial structure but leave the reader alone with just the terms. Either describe these or omit it.
   \item \comment[JR]{will add some description on urban form indicators etc.}
   
  %This also applies to the three segregation indexes: what exactly do they measure?
  \item 
  
  % The description of the spatial generator’s procedure, outlined in paragraph 2.7, would also benefit if more information is added, for example, to delineate the terms “strength of attrac! tion” (attraction in which respect?) and “strength of the diffusion process” (is strength equal to intensity? If so, is it then in space and/or time?). 
  \item \comment[JR]{will add more precision in the description of the generator}

  %While the methodological part of the paper is convincing the theoretical and epistemological background is less so. In short, the paper would improve significantly, if most of the geographical and empirical associations would be left out. Even claimed by the author(s), the approach has not much to do with the empirical growth or development of cities, just because the approach reduces cities’ growth to the growth itself, omitting issues such as historical genesis, cultural differentiation, economic background political and legal circumstances, etc. 
  \item \comment[JR]{most of this empirical literature was added because of one CEUS reviewer, no ?}

  % The problem, from a geographical point of view, is that the paper switches between “geographical” and “spatial” without making any differentiation. In so doing, the author(s) runs immediately into the problem of adequate definitions. In paragraph 1.2 the definition of geographical systems as “social agents interacting with one another, within a limited portion of space” is misleading. First, it is a tautology (geographical systems in space); secondly, geographical cannot be described with social agents, because geographical systems are composed of spatial units, their relations, and a spatially defined system border. Social systems, in turn, might be described with humans as elements, their relations, and again a socially defined system border. Moreover, with respect to sociological system theory (Luhmann 1993), space does not add any knowledge to the nature of social systems. This approach can be (and I would add: has to be) criticised, however, not by interming! ling fuzzily the social with the geographical, as happened with this definition. 
  \item \comment[JR]{by spatial we indeed mean some kind of spatial container for agents right ? which is far from geographical in the sense he puts it indeed}

  % Then, in paragraph 1.3, the complexity of systems with respect to segregation is also problematically described. Emergent behaviour that may lead to segregation is first and foremost a social process, taking social interactions at the local scale (income, housing costs, etc.) and/or at the macro scale (laws, rules, economic system, cultural attitudes, etc.) into account. These interactions can then be associated with the material and relational dimension of geographical spaces. The question remains: what is the role of geographical systems? In this respect, the paper offers some relevant clues such as the MAUP problem or the “uncertain geographic context problem”. 
  \item 

  % The problem of setting space and geography as synonymous continues in paragraphs 1.9 and following. What is described here as “spatial system shape” is the geometry of spaces but not the geography of spaces. 
  % -> yes indeed
  \item 

  % Even more critically is the arbitrary conflation of social and spatial units as happened in paragraph 2.1, as geographical entities are described with “people, housing, networks, etc.” One is inevitably asking oneself: do geographical entities comprise of simply everything? Obviously not, but the paper seems to argue this way. 
  \item 

  % Due to this, attempts to explain the growth of existing cities should be avoided. The approach is reductionist and has its strength in reflecting growth at an abstract level nevertheless. The concluding remark in paragraph 5.1 “With the Schelling experiment, we found … that polycentric and discontinuous cities appear systematically more segregated than compact cities …” is elusive. Either it is a trivial fact, simply because polycentric cities have more centres and thus, by nature, more opportunities from which segregation can originate. Or, the results can be used as a starting point to dig deeper in geographical reasons, taking social, economic, cultural, and political, but also topographical issues into consideration. This geographical reasoning, however, is outside of the scope of this paper. 
  \item \comment[JR]{I rather agree that it is quite reductionist, and that the discussion may be valid only at very abstract levels (on which we were not explicit maybe ?) - maybe make these conclusions more moderate}

   % Formal issues that have to be solved: the numbering of paragraphs is not coherent. The reference of Wheeler (2006) is missing the source of the article.
   \item 

\end{enumerate}


\bigskip
\bigskip

\textbf{Second referee:}
 

%This manuscript presents a methodology to explore the effects of spatial parameters in ABM. They basically explore what they called “generator parameters”, to distinguish them from other statistical ones, which include density spatial distribution and patterns. To do so, they compared phase diagrams and calculated the mean of final values. 
%As case-studies, they used the actual distributions of three European cities as a means to constraint the density generation in two archetypical model (Schelling's segregation model and Epstein and Axtell's Sugarscape). 

\medskip


\begin{enumerate}

  % While the different outcomes of the two implementation exercises are not the main point, I would suggest authors to provide some more elaborated explanations. The two models have different behavioural assumptions and represent a different 'interaction' context. How can be extrapolate and isolate spatial effects when models have different logics and theoretical purposes? 
  \item 

  % In 4.1, the authors suggest potential developments and point to certain limitations. While limitations are well-discussed, readers who are not specialist of spatial models could benefit from a more comprehensive discussion on potential applications and extensions. For instance, it would be useful to have suggestions on how to use this methodology to run empirical models. Suppose to have a nice dataset and a case of a urban context to be modelled. Could the exploration of different spatial configurations even be used to run 'counter-factual' or 'what if' tests on actual spatial configurations? 
  \item \comment[JR]{this is a relevant remark but far beyond the scope of the paper, as an extension of what we did but with perturbation of real datasets / sensitivity to noise - Julien wanted to begin working on that actually and it should be integrated into spatialdata at some point - we also are supposed to implement such a perturbator to test the 4city model}

  % Furthermore, there are now examples of spatial models that go beyond administrative neighborhoods and overimposed spatial boundaries, i.e., egohoods (e.g., https://onlinelibrary.wiley.com/doi/abs/10.1111/1745-9125.12006), whcih could be interesting to see implemented in ABM. Spatial econometrics based on empirical dataset is developing fast. It would be useful to understand how these models can enrich ABM and vice-versa. 
  \item 

  % Furthermore, even among limitations, perhaps authors could discuss data limitations and the fact that social processes (networks, mobility) cannot conflate with spatial dynamics. Here, it is a bit misleading to call "generator parameters" ONLY the spatial ones! 
  \item 

  % In this respect, I would encourage authors to further develop their closing section, which now does not stimulate readers to imagine implications of authors' findings and message to their own research. I would expand this section to provide a more detailed discussion on implications for model makers, experts on social geography and urban scholars who work with ABM, non-specialist in geography who would like to base their models in more proper geographical constraints. And perhaps, authors could suggest future line of research that could show-case their argument. 
  \item 

%I appreciate that models are fully available to readers. 

\end{enumerate}






%\makeletterclosing





\end{document}


%% end of file `template.tex'.