\documentclass[Afour,sageh,times]{sagej}
\usepackage{moreverb,url}

\usepackage[colorlinks,bookmarksopen,bookmarksnumbered,citecolor=red,urlcolor=red]{hyperref}

\usepackage{chngpage}

\newcommand\BibTeX{{\rmfamily B\kern-.05em \textsc{i\kern-.025em b}\kern-.08em
T\kern-.1667em\lower.7ex\hbox{E}\kern-.125emX}}

\def\volumeyear{2017}


\usepackage[utf8]{inputenc}
\usepackage[T1]{fontenc}

%%%%%
% variable to include comments or not in the compilation ; set to 1 to include
\def \draft {1}
%\def \draft {0}



\usepackage{xparse}
\usepackage{ifthen}

\DeclareDocumentCommand{\comment}{o m o o o o}
{\ifthenelse{\draft=1}{
  \IfValueT{#1}{
      \textcolor{red}{\textbf{C (#1) : }#2}
      \IfValueT{#3}{\textcolor{blue}{\textbf{A1 : }#3}}
      \IfValueT{#4}{\textcolor{ForestGreen}{\textbf{A2 : }#4}}
      \IfValueT{#5}{\textcolor{red!50!blue}{\textbf{A3 : }#5}}
      \IfValueT{#6}{\textcolor{Aquamarine}{\textbf{A4 : }#6}}
    }
    \IfNoValueT{#1}{
      \textcolor{red}{\textbf{C : }#2}
      \IfValueT{#3}{\textcolor{blue}{\textbf{A1 : }#3}}
      \IfValueT{#4}{\textcolor{ForestGreen}{\textbf{A2 : }#4}}
      \IfValueT{#5}{\textcolor{red!50!blue}{\textbf{A3 : }#5}}
      \IfValueT{#6}{\textcolor{Aquamarine}{\textbf{A4 : }#6}}
    }
 }{}
}




% todo
%\newcommand{\todo}[1]{
%\ifthenelse{\draft=1}{\textcolor{red!50!blue}{\textbf{TODO : \textit{#1}}}}{}
%}
\DeclareDocumentCommand{\todo}{o m}{
  \ifthenelse{\draft=1}{
    \IfValueT{#1}{\textcolor{red!50!blue}{\textbf{TODO (#1) : \textit{#2}}}}
    \IfNoValueT{#1}{\textcolor{red!50!blue}{\textbf{TODO : \textit{#2}}}}
  }{}
}

%\date{}
%\renewcommand\abstractname{\fontsize{14pt}{0}\textbf{Abstract}\selectfont}

%\usepackage[left=25mm, right=25mm, top=25mm, bottom=25mm, includehead=false, includefoot=false]{geometry}

%\usepackage{graphicx}
%\usepackage{url}
%\usepackage[round,semicolon]{natbib}  % Citation styles https://www.sharelatex.com/learn/Natbib_citation_styles
%\bibliographystyle{humannat}
%\renewcommand{\bibsection}{}
%\renewcommand{\bibhang}{\setlength{-1px}}


%\usepackage{authblk} % For author lists
%\renewcommand\Authfont{\fontsize{11}{1}\selectfont}
%\renewcommand\Affilfont{\fontsize{9}{1}\selectfont}

%\renewcommand*\footnoterule{}

\usepackage[table]{xcolor}
\usepackage[parfill]{parskip} % Line between paragraphs
\usepackage{amsmath}
%\pagenumbering{arabic} 

%\usepackage{sectsty}
%\allsectionsfont{\sffamily}

%\usepackage[pdftex]{hyperref} 
%\hypersetup{pdfborder={0 0 0} }


\usepackage[flushleft]{threeparttable}



%%%%%%%%%%%%%%%%%%%%%%%%%
%% -- from ecrc template
%%%%%%%%%%%%%%%%%%%%%%%%%


%% set the volume if you know. Otherwise `00'
\volume{00}

%% set the starting page if not 1
\firstpage{1}



\jid{}


\biboptions{authoryear}




\usepackage{soul}
\soulregister\cite7
\soulregister\citep7
\soulregister\ref7

%\usepackage[final]{changes}
\usepackage{changes}


\setaddedmarkup{\textcolor{black}{\hl{#1}}}
\setdeletedmarkup{\textcolor{red}{\sout{#1}}}











\begin{document}

% **************  TITLE AND AUTHOR INFORMATION **************


\runninghead{Raimbault \& al.}


\title{Initial spatial conditions in simulation models: the missing leg of sensitivity analyses?}

\author{Juste Raimbault\affilnum{1,2}, 
Cl{\' e}mentine Cottineau\affilnum{3},
Marion Le Texier\affilnum{4}, 
Florent Le N{\' e}chet\affilnum{2}, 
Romain Reuillon\affilnum{1,5}
}

\affiliation{\affilnum{1}UMR 8504 G{\'e}ographie-cit{\'e}s, Paris, France\\
\affilnum{2}Laboratoire Ville Mobilit{\'e} Transport, Universit{\'e} Paris-Est, France\\
\affilnum{3}Centre for Advanced Spatial Analysis, University College London, UK\\
\affilnum{4}UMR 6266 IDEES, Universit{\'e} de Rouen Normandie, France\\
\affilnum{5}Institut des Syst{\`e}mes Complexes Paris Ile-de-France, France}

\corrauth{Corresponding author
address.}

\email{c.cottineau@ucl.ac.uk}



% **************  ABSTRACT  **************

\begin{abstract}
%\noindent
%\setlength{\parindent}{0pt}
Although simulation models of geographical systems in general and agent-based models in particular represent a fantastic opportunity to explore socio-spatial behaviours and to test a variety of scenarios for public policy, the validity of generative models is uncertain until their results are proven robust. Sensitivity analysis usually include the analysis of the effect of stochasticity on the variability of results, as well as the effects of small parameter changes. However, initial spatial conditions are usually taken for granted in geographical models, thus leaving completely unexplored the effect of spatial arrangements on the interaction of agents and of their interactions with the environment. In this contribution, we present a method to assess the effect of initial spatial conditions on simulation models, using a systematic generator controlled by meta-parameter to create density grids used in spatial simulation models. We show, with the example of two very classical agent-based models (Schelling's models of segregation and Sugarscape) that the effect of space in simulation is significant, and sometimes even larger than parameters themselves. We do so using high performance computing in a very simple and straightforward open-source workflow.
\end{abstract}

\keywords{Space, Initial conditions, Sensitivity, ABM}

\maketitle


% **************  MAIN BODY OF THE PAPER **************


%%%%%%%%%%%%%%%%%%%%%%
\section{Introduction}
%%%%%%%%%%%%%%%%%%%%%%

Simulation has been recognised and increasingly used by geographers to explore various geographical processes and problems within virtual laboratories \citep{Quesneletal2009} \comment{Batty 1978? James Doran 1970's : lire thèse de Seb ;) }. It appears as a very fruitful way to overcome the difficulty of analytic resolution of many spatial models developed in the past, and to explore the possible trajectories of social and ecological systems in space and time. The specificity of geographical models compared with models from other social sciences is usually regarded as the way geographers consider space and spatial interactions, driven by an explicit interest in the way space influences the outcomes of the model. Geographers are indeed concerned about understanding and modelling how space plays a role in social interactions and environmental processes, and whether its action is placed-based or place-neutral. We think simulation can become a very good tool to achieve this, provided that models include relevant spatial description and modelling, behavioural rules that take space into account, and provided model evaluation stresses the sensitivity of output variations to the way space is modelled. This paper aims to fill a methodological and conceptual gap, which is a systematic testing of the sensitivity of model's outcomes to initial spatial conditions. To demonstrate the genericity of our approach, we develop two applications from simulation models which are commonly used as case studies for comparing and aligning simulation models \citep{Axtelletal1996}: Schelling's model of segregation and Epstein and Axtell's Sugarscape model. Both models are defined at the intraurban level.

An urban system consists of a large number of social agents that interact with each other at a microscopic level in a system-time that evolves irreversibly creating temporal and cumulative effects. Self-organization has been shown to be a central feature of urban systems~\citep{AllenSanglier1981,Portugali2000}, and emergent properties at qualitatively differentiated scales~\citep{AzizAlaouiBertelle2009} must be obtained through simulation methods~\citep{Wu2002,Batty2007}. In that frame, complexity is partially due to bifurcations \citep{Wilson1981} which are determinant in spatial systems~\citep{Wilson2002}. 


In spatially explicit simulation models, the non-linearity of local interactions is therefore very likely to sublimate small perturbations in the initial spatial setting, making it difficult to interprete the resulting global structures. \comment[JR]{je suis pour garder le passage sur la non-ergodicité (commenté ci-après) ; ça vous parait trop abstrait ?}\comment[MLT]{Non, mais je pense que la seconde phrase pourrait être reformulée/développée pour mieux raccrocher avec le reste, qu'en penses-tu?} A crucial aspect of most spatio-temporal complex systems is their non-ergodicity~(\cite{pumain2012urban}) (the property that cross-sectional samples in space are not equivalent to samples in time to compute statistics such as averages), what witnesses generally strong spatio-temporal path-dependencies in their trajectories. Similar to what Gell-Mann calls \emph{frozen accidents} in any complex system~\cite{gell1995quark}, a given configuration contains clues on past bifurcations, that can have had dramatic effects on the state of the system. 


At initialisation, we then expect the spatial distribution of agents to influence results in the long run \cite{Castellanoetal2009}, as agents' rule of action itself might depend on the spatial structure of the environment: households can have different preferences with respect to the built-environment (\cite{SpielmanHarrison2014}), mechanisms of surrounding sensing are impacted by different distributions of density (\cite{Banos2012}) or the scale of modelled environments (\cite{LauriJaggi2003, FossettDietrich2009}), etc. The way modellers abstract space is therefore a central element of any spatially explicite simulation model. Although this may seem obvious, cities are not regular grids of isotropic densities. However, the uniform grid which represents space in most simulation models is potentially not enough to represent urban processes because density and accessibility have environmental, economic and social consequences. An intermediate and more meaningful way of abstracting space might thus be to consider, not the peculiarities of every city, but their broad density structures. In Europe for example, one can find broad types of density distributions (\cite{LeNechet2015}). 

\subsection{Objective}

In this paper, we suggest tackling sensitivity to spatial conditions by generating a variety of density grids to feed into simulation models at initialisation, and exploring the sensitivity of the model outcomes to the variation of meta-parameters (i.e. parameters used to generate initial spatial conditions). The purpose is two-fold: 1/ to test the robustness of simulation results to small variations of meta-parameters within a typical category of space (a monocentric case for example\comment[FL]{je ne trouve pas cela facile à comprendre car on n'a pas encore parlé des différentes formes de villes}) and 2/ to study the non-trivial effects of typical categories of spatial distribution (monocentric \textit{vs.} polycentric for example) on the results of a given model.

%%%%%%%%%%%%%%%%%%%%%%
\subsection{Previous considerations of urban form in simulation models}

\begin{itemize}
\item spatial context
\begin{itemize}
\item extent
\item objects
\item shape
\end{itemize}

\item spatial "encoding"
\begin{itemize}
\item heterogeneity
\item algorithm of disaggregation
\item uncertainty
\end{itemize}

\item spatial effects
\begin{itemize}
\item topology
\item interaction (??)
\end{itemize}

\item spatial outputs
\begin{itemize}
\item aggregation
\item use of models
\end{itemize}
\end{itemize}
%\comment[FL]{Je pense que cette partie est vraiment utile, et devrait être étoffée, en procédant peut être à une typologie plus systématique des "initial condition" qui nous intéressent. Il y a déjà une différence entre l'hétérogénéité de l'espace (densités réparties mono ou poly, etc) et les différentes façons de représenter l'espace (grilles carrées, hexagones). Par ailleurs aussi des discussions sur l'incertitude de l'information (est ce que par exemple les emplois downscalés au carreau INSEE par Pivano 2016 c'est vraiment fiable ou très ?) et la précision de la grille} 

The effect of the spatial configuration on area-based attributes of human behaviours has been largely discussed in geostatistics, meanly since the exposure of the Modifiable Areal Unit Problem (MAUP) \citep{Openshaw1984,FotheringhamWong1991}. Recently, \citet{Kwan2012} claims for a careful examination of what she coins the uncertain geographic context problem (UGCoP), that is of the spatial configuration of geographical units even if the size and delineation of the area are the same. Considerations of such issues in the geographic simulation model literature are rather scarce, but they have been some noticeable examples of papers attempting to evaluate the effects of the same set of mechanisms runned from different initial spatial conditions. Mainly, these approaches have been focusing on the impact of geo-localised input data accuracy, of the spatial system shape, precision, and boundaries, as well as of the impact of spatial heterogeneity.

\textbf{Geo-localised input data accuracy:} \citep{Thomasetal2017} show that data selection in LUTI model is interelated to the delineation of the spatial sytem boundaries and scale of analysis. They provide a few examples of how the use of Exploratory Spatial Data Analysis (ESDA) prior to simulation runs help avoiding measurement errors of model behaviour and outcomes.

\textbf{Spatial system shape, precision, and boundaries:} \citet{FlacheHegselmann2001} show that chances for random emergence of a stable cluster of similar agents in a Schelling-like model are highest in a rectangular grid and lowest in a hexagonal grid and that an irregular (Voronoi-diagram) city lattice structure favours migration stabilisation around decentralised clusters of similar agents. \citet{Banos2012} compares the behaviour of Schelling segregation model on city lattices formalized as either grid, random, scale-free and Sierpinski networks and concludes that the presence of cliques in graph-based urban structures favor segregationist behaviours. \citet{LeTexierCaruso2017} simulate an individual-based dynamic model where agents move and exchange coins across regions. Using a set of different theoretical spatial systems, they demonstrate the impact of the regularity and aggregation levels, or centrality/periphery effects, on spatial diffusion dynamics.

\textbf{Spatial heterogeneity:} \citet{StaufferSolomon2007} introduce asymetric interactions and empty residences in Schelling's model run on a large and regular lattice. They reveal conjoint and non-linear effects on the vacancy rates and tolerance levels on segregation patterns. \citet{Gauvinetal2010} run Schelling's segregation process in an open city-lattice to study how the variations in tolerance levels, vacancy rates and city attractiveness may create lines of vacancy lots between clusters of agents. They conclude on the functional role of vacancies, which allow weakly tolerant agents to live and be satisfied in a city environment they nevertheless perceive as hostile. \citet{HatnaBenenson2012} show that their model replications run on a 50x50 torus with 2\% of empty cells were not sensitive to the initial patterns (random and fully segregated distribution of agents).

\textbf{Combination:} \citet{Singhetal2009} show that the segregation patterns obtained by Schelling for certain tolerance values are strictly a small city phenomenon (8x8 city-lattice) and do not work for a larger spatial lattice (100x100), where segregation appears only for certain combinaisons of tolerance threshold and vacancy density values. 

In this paper, we build on the contributions of these previous studies and go further by systematically measuring the impact of the city lattice on model behaviour so far only assessed through a small set of illustrative spatial structures. We illustrate the potential generalization of our approach by running two distinctive agent-based models: Schelling's model of residential segregation and Sugarscape.

%The example of transportation networks is a good illustration, as their spatial shape and hierarchy is strongly influenced by past investment decisions, technical choices, or political decisions sometimes not rational~(\cite{zembri2010new}). Some aggregated indicators will not take into account positions and trajectories of each agent (such as segregation in the Schelling model) but others, as in the case of spatial patterns of accessibility in a system of cities, fully capture the path-dependency and may therefore be highly dependent of the initial spatial configuration. It is not clear for example what shifted the economical and political capital of France from Lyon to Paris in the early Middle Age, some assumptions being the reconfiguration of trade patterns from South to North of Europe and thus an increased centrality for Paris due to its spatial position: the bifurcation induced by socio-economic and political factors took a deep significance with worldwide repercussions until today when magnified by the spatial configuration. \comment[JR]{quelqu'un sait si d'une part c'est vrai et d'autre part si des references backupent ca ? - je sors ca completement du chapeau pour l'instant mais ça me semble intuitivement raisonnable et une bonne illustration.}[(MLT)L'idée de donner un exemple me semble bonne mais je ne suis pas sûre qu'on l'amène de la bonne manière? --> en fait je crois que l'on a pas besoin de ce paragraphe illustratif, ça complique le papier. Qu'en pensez-vous?]  [(FL) : illustrer par les transports c'est une bonne idée mais pas besoin de parler du moyen âge! la faible croissance de Dijon depuis 30 ans peut être une illustration tarte à la crême. mais pas sûr effectivement qu'on ait absolument besoin d'illustration, en tous cas il n'y en a pas dans le § d'avant (bientôt après ;o)]

%%%%%%%%%%%%%%%%%%%%%%
\section{Methods}
%%%%%%%%%%%%%%%%%%%%%%

In this section, we detail the method developed to analyse the sensitivity of simulation models to initial spatial conditions. The general method workflow is illustrated in Figure \ref{fig:method}. In addition to the usual protocol (upper branch in figure \ref{fig:method}, which consists of running a model $\mu$ with various values of its parameters and relating these variations of values to the variations in the simulation results, we here introduce a spatial generator (lower branch in figure \ref{fig:method}), which itself is determined by input parameters and produces sets of spatial initial conditions. Initial spatial conditions are clustered to represent types of spaces ex-ante (for example: monocentric or polycentric density grids), and the sensitivity analysis of the model is now run against $\mu$ parameters as well as spatial parameters or spatial types. It allows the sensitivity analysis to produce qualitative conclusions regarding the influence of spatial distribution on the outputs of simulation models, alongside the classic variation of parameter values.

%%%%%%%%%%%%%%%%%%%%%%
\begin{figure*}[htbp] \begin{center} 
\resizebox{0.9\textwidth}{!}{ 
	\includegraphics{figures/SchemaMeta_1.png}
} \caption{General workflow of our method} \label{fig:method} \end{center} \end{figure*} %
%%%%%%%%%%%%%%%%%%%%%%

%%%%%%%%%%%%%%%%%%%%%%
\subsection{Constructing a spatial generator}

Our spatial generator applies an urban morphogenesis model (\cite{Batty2007}), which has been generalised, explored and calibrated (\cite{Raimbault2014}). In a nutshell, grids are generated through an iterative process which, starting from a void grid, adds a quantity $N$ (population) at each time step $t$, allocating it through preferential attachment \comment[FL]{à quoi?} characterised by its strength of attraction $\alpha$. More precisely, each added unit has a probability equal to $P_i^{\alpha}/\sum_k P_k^{\alpha}$ to be added to patch $i$ with population $P_i$, all $N$ units being added independently and in parallel. At the end of each time step, this growth process is smoothed $n$ times using a diffusion process of strength $\beta$: each patch transmits an equal share of $\beta\cdot P_i$ to its extended neighborhood (8 surrounding patches). To avoid border effects such as a reflexion on the border of the world, border patches diffuse to the outside\comment[FL]{comment est-ce qu'on contrôle la perte de population induite?}. The model stops when a fixed total population \comment{n?} is reached. Grids are thus generated from the combination of the values of these four meta-parameters $\alpha$, $\beta$, $n$ and $N$, in addition to the random seed. To ease our exploration, only the distribution of density is allowed to vary rather than the size of the grid, which we fix to a 50x50 square environment of 100,000 units (cf. figure \ref{fig:spatialGen}). Typical value ranges for the other parameters are $\alpha\in\left[0.5,4.0\right]$, $\beta \in\left[0,0.3\right] $, $N\in \left[100,10000\right]$, $n\in\left[1,4\right]$.
\comment{A detailler en incluant les equations de morphogenese}[(JR) j'ai détaillé le processus, c'est suffisant ?]\comment[MLT]{Il me semble que l'on pourrait mieux expliciter la relation N/n (et peut-être du coup inverser N et n). Plus il me semble que l'on devrait dire plus clairement à quoi se réfère alpha : à une cellule $i$? au pas de temps $t$? Pourquoi ne pas indexer alpha? Enfin, je crois que l'on gagnerait encore plus en clarté si on expliquait formellement comment tous les paramètres sont reliés entre eux, notamment pour expliquer l'arrêt de la procédure d'attribution d'une nouvelle pop à la grille.} \comment[FL]{je crois que N c'est l'incrément de pop et n le nombre de fois où c'est smoothé mais c'est quoi la pop totale?}



%%%%%%%%%%%%%%%%%%%%%%
\begin{figure*}[htbp] \begin{center} 
\resizebox{0.9\textwidth}{!}{ 
	\includegraphics{figures/spatialGen.png}
} \caption{Four examples of grids produced by the spatial generator. The lighter the red, the denser the area. Changing the growth rate $N$ allows to have more or less chaotic shapes (two first compared to the two last grids for example) corresponding to different levels of convergence of the model, whereas local radius can be tuned with the interplay of aggregation strength $\alpha$ and diffusion strength $\beta$.} \label{fig:spatialGen} \end{center} \end{figure*} %
%%%%%%%%%%%%%%%%%%%%%%


%%%%%%%%%%%%%%%%%%%%%%
\subsection{Including a (typical) variety of initial conditions in the sensitivity analysis}

In order to generate density grids which correspond to empirical density distributions, we select among the generated grids using an objective function which matches the point cloud of 110 metropolitan areas in Europe described by four dimensions of spatial structure : their concentration index, hierarchy index, centrality index and continuity index (cf. \cite{LeNechet2015}). A stochastic exploration of a Latin Hypercube Sampling of 2000 points in the 4-dimensional space of parameters {$\alpha$, $\beta$, $n$, $N$} gives a subset of 170 interesting grids matching empirical densities, which constituted our set of different initial spatial conditions. These are further clustered into three classes of morphology: compact (e.g. Vienna), polycentric (Liege) and discontinuous (Augsburg) in order to evaluate the non-trivial effects of urban form on simulation results. We select 15 grids of each type to work with.

%cite{Gauvinetal2009}


%%%%%%%%%%%%%%%%%%%%%%
\subsection{Comparing Phase Diagrams}

In order to test for the influence of spatial initial conditions on model outputs, we need a systematic method to compare phase diagrams. Indeed, we have as many phase diagrams than we have spatial grids, what makes a qualitative visual comparison not realistic. A solution is to use systematic quantitative procedures. To our knowledge there exist no well established method to compare phase diagrams in agent-based modeling and geosimulation literature. Several potential methods from other fields such as environmental science could be used, but we keep it simple and such methodological considerations are furthermore auxiliary to the main purpose of this paper. We propose therefore an intuitive measure corresponding to the share of between-diagrams variability relative to their internal variability. More formally, the distance is given by

\begin{equation}\label{eq:phase-distance}
d_r\left(\alpha_1,\alpha_2\right) = 2 \cdot \frac{d(f_{\vec{\alpha_1}},f_{\vec{\alpha_2}})^2}{Var\left[f_{\vec{\alpha_1}}\right] + Var\left[f_{\vec{\alpha_2}}\right]}
\end{equation}

where $f_{\vec{\alpha}}(\vec{x})$ is the phase diagrams on parameters $\vec{x}$ at given meta-parameters $\vec{\alpha}$. $d$ is a functional distance that we take simply as Euclidian distance. The variances are estimated within each phase diagram $f_{\vec{\alpha_i}}$.

\comment[JR]{j'ai simplifié au max}\comment[mLT]{Attention, on utilise alpha plus haut.}

%%%%%%%%%%%%%%%%%%%%%%
\subsection{Model Exploration Workflows as an meta-sensitivity analysis method}

The last methodological point which we need to emphasis is the relation between the workflow we introduce and model exploration workflows. The ideas of multi-modeling and extensive model exploration are nothing from new as Openshaw already advocated for ``model-crunching'' in, but their effective use only begins to emerge thanks to the apparition of new methods and tools together with an explosion of computation capabilities. The model exploration platform OpenMole~\citep{reuillon2013openmole} allows to embed any model as a blackbox, write modulable exploration workflow using advanced methodologies such as genetic algorithms and distribute transparently the computation on large scale computation infrastructures such as clusters or computation grids. In our case, the workflow tool is a powerful way to embed both the sensitivity analysis and the meta-sensitivity analysis, and allow to couple any generator with any model in a straightforward way as soon as the model can take its spatial configuration as input or from an input file. In this paper, we use the OpenMole platform for spatial environment and model coupling, placing ourselves in the renewed framework of multi-modeling claimed by \cite{cottineau2016back}.

%%%%%%%%%%%%%%%%%%%%%%
\section{Application cases}
%%%%%%%%%%%%%%%%%%%%%%

In this section, we briefly recall the main components of the two ``classical'' agent-based simulation models used to test how spatial density variations may impact simulation models behaviour and results, and how general the method proposed is.

%%%%%%%%%%%%%%%%%%%%%%
\subsection{Schelling's model of residential segregation}

Schelling's model consists in an abstract urban housing market where agents of different nature sense their environment, evaluate their satisfaction in terms of neighbourhood composition, and relocate if unsatisfied. It has been shown by \cite{Schelling1969} that even tolerant agents tended to produce segregated patterns because of the complexity of their local interactions. The main parameters of this model are the tolerance level (\% of agents similar to {\it ego}), the scope of sensing and the percentage of vacant spaces in the housing market. In addition, we are interested in testing the impact of the spatial distribution of housing capacity in this project, using the generated grids. \comment{We extend the implementation of... The outcome of the model is measured as a phase diagram of segregation level (Dissimilarity, Moran's I, Entropy, Exposure, all four of them?). .}

%%%%%%%%%%%%%%%%%%%%%%
\subsection{Sugarscape model of resource extraction and population settlement}

Sugarscape is a model of resource extraction which simulates the unequal distribution of wealth within a heterogenous population (\cite{EpsteinAxtell1996}). Agents of different vision scopes and different metabolisms harvest a self-regenerating resource available heterogeneously in the initial landscape, they settle and collect this resource, which leads some of them to survive and others to perish. The main parameters of this model are the number of agents, their minimal and maximal resource. In addition, we are interested in testing the impact of the spatial distribution of the resource in this project, using the generated grids. We extend the implementation with agents wealth distribution of~\cite{li2009netlogo}. The outcome of the model is measured as a phase diagram of an index of inequality for ressource distribution (Gini index). 


%%%%%%%%%%%%%%%%%%%%%%
\section{Results}
%%%%%%%%%%%%%%%%%%%%%%

\comment[FL]{il faudrait peut être davantage renforcer l'analyse a propos de figure 1 de si c'est plutôt les mu parameters ou les spatial parameters qui ont une grosse influence sur les résultats.}[(JR) assez explicite pour sugarscape, j'ai rajouté un peu tout de même.]



%%%%%%%%%%%%%%%%%%%%%%
\subsection{Schelling's model}

We proceed to 4,500,000 simulation runs of the Schelling model (1000 parameter combinations x 45 density grids x 100 replications), using OpenMOLE to distribute the computation, and apply segregation measures to characterise the results. We find that the different urban morphologies impact the parameter interaction patterns, and that polycentric and discontinuous cities appear systematically more segregated than compact cities in terms of dissimilarity and entropy index.



%%%%%%%%%%%%%%%%%%%%%%
\subsection{Sugarscape model}

For Sugarscape, 2,500,000 simulations (1000 parameter points x 50 density grids x 50 replications) allow us to show that the model is more sensitive to space than to its other parameters, both qualitatively and quantitatively: the amplitude of variations across density grids is larger than the amplitude in each phase diagram, and the behavior of phase diagram is qualitatively different in different regions of the morphological space.

More precisely, we explore a grid of a basic parameter space of the model, which three dimensions are the population of agents $P\in \left[10;510\right]$, the minimal initial agent ressource $s_{-}\in \left[10;100\right]$ and the maximal initial agent ressource $s_{+}\in \left[110;200\right]$. Each parameter is binned into 10 values, giving 1000 parameter points. We run 50 repetitions for each configuration, what yield reasonable convergence properties. The initial spatial configuration varies across 50 different grids, generated by sampling meta-parameters for the generator in a LHS. We did not use the clustered grids to test the flexibility of our framework, which is demonstrated in this case by a direct sequential coupling of the generator and the model. We mesure the distance of all 3-dimensional phase diagrams to the reference phase diagram computed on the default model setup (see Fig.~\ref{fig:sugarscape-distance} for its morphological positioning regarded generated grids), using equation~\ref{eq:phase-distance} with the L2 distance to ensure direct interpretability. Indeed, it gives in that case the average squared distance between corresponding points of the phase diagrams, relative to the average of the variance of each. Therefore, values greater than 1 will mean that inter-diagram variability is more important than intra-diagram variability.

% summary stats
%   Min. 1st Qu.  Median    Mean 3rd Qu.    Max. 
% 0.08909 0.19790 1.52200 1.29600 2.16400 2.98100 

We obtain a very strong sensitivity to initial conditions, as the distribution of the relative distance to reference across grids ranges from 0.09 to 2.98 with a median of 1.52 and an average of 1.30. It means that in average, the model is more sensitive to meta-parameters than to parameters, and the relation variation can reach a factor of 3. We plot in Fig.~\ref{fig:sugarscape-distance} their distribution in a morphological space. The reduced morphological space is obtained by computing 4 raw indicators of urban form, namely Moran index, average distance, rank-size slope and entropy (see~\cite{LeNechet2015} for precise definition and contextualization), and by reducing the dimension with a principal component analysis for which we keep the first two components (92\% of cumulated variance). The first measures a ``level of sprawl'' and of scattering, whereas the second measures aggregation.\footnote{We have $PC1 = 0.76\cdot distance + 0.60\cdot entropy + 0.03\cdot moran + 0.24\cdot slope$ and $PC2 = -0.26\cdot distance + 0.18\cdot entropy + 0.91\cdot moran + 0.26\cdot slope$.} We find that grids producing the highest deviations are the ones with a low level of sprawl and a high aggregation. It is confirmed by the behavior as a function of meta-parameters, as high values of $\alpha$ also yield high distance. In terms of model processes, it shows that congestion mechanisms induce rapidly higher levels of inequality. To put these results in perspective of our workflow given in Fig.~\ref{fig:method}, we have a sensitivity to spatial parameters in average greater than the sensitivity to model parameters.


% pca of morphological space
% "","PC1","PC2","PC3","PC4"
%"distance",0.762358566609464,-0.260991693298744,0.200656405132039,0.557162237616392
%"entropy",0.601306167355116,0.181706245959277,0.0958379422351422,-0.772158547261002
%"moran",0.0311129390452153,0.912155429075071,0.30114271129527,0.276256268103684
%"slope",0.237217819823539,0.258531718397015,-0.927289147645628,0.130475642169329

%%%%%%%%%%%%%
\begin{figure}
\centering
\includegraphics[width=0.5\textwidth]{figures/relativedistance_metaparams}\\
\includegraphics[width=0.5\textwidth]{figures/relativedistance_morphspace}
\caption{\textbf{Relative distances of phase diagrams to the reference across grids.} (Top) Relative distance as a function of meta-parameters $\alpha$ (strength of preferential attachment) and diffusion ($\beta$, strength of diffusion process). (Bottom) Relative distance as a function of two first principal components of the morphological space (see text). Red point correspond to the reference spatial configuration. Green frame and blue frame give respectively the first and second particular phase diagrams shown in Fig.~\ref{fig:sugarscape-phasediagrams}.}
\label{fig:sugarscape-distance}
\end{figure}
%%%%%%%%%%%%%



% phase diagrams -> ok well different qualitatively
%          spAlpha spDiffsteps spDiffusion spGrowth spPopulation
% id=27 : 0.7913103    2.376837   0.1440293 157.4147 4852.746
% id=0 : 2.562398    3.753032   0.1316788 128.4632 4753.983
% maxSugar = 110


%%%%%%%%%%%%%
\begin{figure}
\centering
\includegraphics[width=0.24\textwidth]{figures/phasediagram_id27_maxSugar110}
\includegraphics[width=0.24\textwidth]{figures/phasediagram_id0_maxSugar110}
\caption{\textbf{Examples of phase diagrams.} We show two dimensional phase diagrams on $(P,s_-)$, both at fixed $s_+ = 110$. (Left) Green frame, obtained with $\alpha = 0.79$, $n=2$, $\beta = 0.14$, $N=157$; (Right) Blue frame, obtained with $\alpha = 2.56$, $n=3$, $\beta = 0.13$, $N=128$.}
\label{fig:sugarscape-phasediagrams}
\end{figure}
%%%%%%%%%%%%%

We now check the sensitivity in terms of qualitative behavior of phase diagrams. We show the phase diagrams for two very opposite morphologies in term of sprawling, but controlling for aggregation with the same $PC2$ value. These correspond to the green and blue frames in Fig.~\ref{fig:sugarscape-distance}. The behaviors are rather stable for varying $s_+$, what means that the poorest agents have a determinant role in trajectories. The two examples have not only a very distant baseline inequality (the ceil of the first 0.35 is roughly the floor of the second 0.3), but their qualitative behavior is also radically opposite: the sprawled configuration gives inequalities decreasing as population decreases and decreasing as minimal wealth increases, whereas the concentrated one gives inequalities strongly increasing as population decreases and also decreasing with minimal weights but significantly only for large population values. The process is thus completely inverted, what would have significant impacts if one tried to schematize policies from this model. This second example confirms thus the importance of sensitivity of simulation models to the initial spatial conditions.

\comment{Ajouter un paragraphe sur le lien entre forme de grille et comportement du modèle/résultats.}

%%%%%%%%%%%%%%%%%%%%%%
\section{Discussion}
%%%%%%%%%%%%%%%%%%%%%%



\comment[FL]{NetLogo qui influence la formalisation de l'espace dans les modèles alors que formalisation de ce dernier peut être plus poussée dans la description du modèle original.}[(JR) bonne idée, y'a pas mal d'exemples ou il faut pas mal contourner l'usage ``naturel'' de Netlogo pour s'en sortir - sur les représentations raster/vecteur c'est souvent un dilemne et ça peut induire des biais d'implémentation - je vais essayer de trouver un exemple simple (ceux que j'ai là trop usine à gaz, exemple Lutecia)]

\comment[MLT]{Au-delà de l'outil NetLogo, parler de la formalisation de l'espace par les physiciens qui a influencé les modèles en géographie (cf: commentaire de Florent que je propose de bouger ici)?}

\comment[FLN]{opportunity for social science to emancipate from the very strong hypotheses of physicists that have become standards (homogeneity and isotropy of space), even though they are never valid in social systems.} 

\comment[CC]{Lancer des pistes sur comment travailler l'effet de la grille pour d'autres types de modélisation type modèles appliqués.}

\comment[MLT]{Mise à disposition des grilles générées? D'un échantillon?}

\paragraph{Comparing phase diagrams}

Comparing phase diagrams is as we saw not straightforward, and further developments of our method imply testing alternative methods for this particular point. For example in the case of the Schelling model, an anisotropic spatial segregation index (giving the number of clusters found and in which region in the parameter spaces they are roughly situated) would differentiate strong \emph{meta phase transitions} (phase transitions in the space of meta parameters). The use of metrics comparing spatial distributions, such as the Earth Movers Distance which is used for example in Computer Vision to compare probability distributions~\cite{rubner2000earth}, or the comparison of aggregated transition matrices of the dynamic associated to the potential described by each distribution, would also be potential tools. Map comparison methods, popular in environmental sciences \comment[MLT]{Et pas en géo? Qu'est-ce que ces méthodes ont de particulier?}, provide numeral tools to compare two dimensional fields~\cite{visser2006map}. To compare a spatial field evolving in time, elaborated methods such as Empirical Orthogonal Functions that isolates temporal from spatial variations, would be applicable in our case by taking time as a parameter dimension, but these have been shown to perform similarly to direct visual inspection when averaged over a crowdsourcing~\cite{10.1371/journal.pone.0178165}. Note that with the intuitive index we used, more subtle distances such as the Earth's Mover Distance can be used.


% \hfill \break
% \itshape{This is a sub, subheading}\normalfont

% \hfill\break

%
%
%\begin{table}[htp]
%
%\begin{center}
%\begin{tabular}{c c c c}
%\arrayrulecolor{black}
%\hline 
%This & Is & A & Table\\
%\arrayrulecolor{lightgray}
%\hline 
%\arrayrulecolor{black}
%Label & 0.1 & 0.2 & 0.3\\
%Label & 1.0 & 2.0 & 3.0\\
%\hline
%\end{tabular}
%\end{center}
%\label{first_table}
%\caption{This is a table caption}
%\end{table}%
%
%
%
%\begin{equation}
%a^2 + b^2 = c^2
%\tag*{Equation 1}
%\end{equation}

%%%%%%%%%%%%%
% Acknowledgements

\begin{acks}
The authors acknowledge the funding of their institutions and the EPSRC project number EP/M023583/1. Results obtained in this paper were computed on the vo.complex-system.eu virtual organization of the European Grid Infrastructure ( http://www.egi.eu ). We thank the European Grid Infrastructure and its supporting National Grid Initiatives (France-Grilles in particular) for providing the technical support and infrastructure.
\end{acks}




%%%%%%%%%%%%
%% References
%%%%%%%%%%%%


\bibliographystyle{SageH}
\bibliography{spacematters,biblio}


\end{document}
