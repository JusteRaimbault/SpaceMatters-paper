%\todo{Response to reviewers
%\begin{itemize}
%\item For reviewer 1, explicitly mention the simulation framework at the beginning (and position it regarding others such as dynamical systems) with literature
%\item Similarly, clarify the aggregation step, how the indicators are macro, have a thematic meaning, and distribution converge clearly (possibly : add Sharpe ratios statistics and type of distrib for better fits)
%\item Rev 1 : (complementary to above) : positioning regarding some analytical works / cite some ? Precise that we do not aim at understanding fully model behavior, but finding some sensitivity to space
%\item Rev 2 : precise which Schelling we use.
%\item On real configs that can not be modified : remark by Marion on MAUP etc ; and add in discussion a layus on synthetic data resembling real configs etc.
%\item Remove or adapt the concept of validation to avoid the epistemological trap
%\item To be debated : do we keep the detailed description of the spatial generator ? (knowing that it is otherwise fully described and explored here : \url{https://arxiv.org/abs/1708.06743})
%\end{itemize}
%}





% ********************************************
% Review EPB :

%Referee: 1

%Comments to the Author

%One sentence in the abstract of the paper, which is a part of the review invitation, was indeed confusing “However, initial spatial conditions are usually taken for granted in geographical models, thus leaving completely unexplored the effect of spatial arrangements on the interactions of agents and of that of agents with their environment.” What can be that special in initial conditions? There are standard notions of equilibrium and, more general “attractor” of the dynamic system and there are “domains” of “basins” of attraction for each attractor that defines to which of the attractors the solution will converge in time (see http://www.scholarpedia.org/article/Basin_of_attraction). May “spatial” in “spatial conditions” make a difference?
%I believe that after reading the paper I understand the way of author(s) thinking. Regrettably, their view of the dynamic system as well as their use of the basic notions and examples, are wrong. In what follows I will try to point to the major steps where the authors deviate, further and further away, from the proper view of the complex dynamic system theory.
%The introduction reads very well while raises some general questions. E.g., cities’ “non-ergodicity” is a kind of a philosophical statements, some urban processes, as buildup area dynamics, can be indeed considered as path-dependent, while some, as urban traffic, are most probably, not. Whatever, introduction does not cause problems.
%Then the methodical section comes, posing the first red flag. I mean the choice of the Schelling and Sugarscape modes as the examples. On the one hand, none of these models deals directly with the physical structure of space; even more, studies of the e.g. Schelling model on graphs show that its dynamics do depend on the physical structure of space. On the other, each of these models has many versions and, as dynamic spatial systems, these versions are very different. The dynamics of some versions of the Schelling model are essentially history-dependent, while the dynamics of the others are not. The objectives of the paper demand very careful recognition between the system with path-dependent and path-independent dynamics.
%From this point on, the paper took a “believe me” path. The impression of the reader is that the authors are absolutely sure in their views and, thus, ignore numerous critical points that may cause reader’s doubts. Even definitions became problematic. “We define a phase diagram as the vector of model outputs considered as a function of model parameters.” Sorry, phase diagram is a standard notion and it is something else, see https://en.wikipedia.org/wiki/Phase_diagram. Further on, “Technically, because of stochasticity, we represent the output of the model for a given combination of parameter values P as the median of the final values of an output indicator O obtained for the replications of the model initialized with P”. Well, the reader can guess now that the intention of the author is to investigate the dynamics of the aggregate characteristics of the output as dependent on the parameter(s) P of the system. Putting aside wrong use of “phase diagram,” this is manageable, while quite standard. However, the part of this statement raises deeper concerns: The dependence of the “median O” on P may work well in case when the stochastic system converges to a globally stable stochastic equilibrium. But what if it doesn’t? Let’s consider the simplest case of a stochastic system with two attractors, each being locally stable stochastic equilibrium. In this case, initial conditions that are close to the border between equilibrium’s basins would evidently result in very different values of O. To avoid a vicious circle, one cannot use median or any other aggregate characteristic before the attractors are properly recognized. That is, one cannot use median before system’s properties are known well enough.
%Ok, let’s skip “median” and follow the paper based on the single solutions. Let us essentially reduce the class of systems to keep in mind too, assuming that (1) for every P, every attractor is just an equilibrium maybe, stochastic one; (2) variation in P can result in variation in the number of equilibriums and their location in a phase space. In this case, conceptually, given K output characteristics, we can indeed recognize the basins of attraction based on the distance between solutions as far as they stabilize in time. Formula (1) is ok in this respect, while I would consider multidimensional O and standard measures, e.g.,   https://math.stackexchange.com/questions/1358633/malalanobis-distance-between-two-multivariate-gaussian-distributions.
%From this point, let us continue with the Schelling model only. Sugarscape is more complex and less studied. In respect to the papers goals, the background property of the Schelling model was formulated in Vinkovic, Kirman, 2006, A physical analogue of the Schelling model, Proceedings of the National Academy of Sciences, 103(51), 19261–19265. Namely, they distinguish between models, which rules enable relocation to a better location only and the rules that enable relocation to a cell of the same utility. Models of the first subclass always stall, with the non-zero (and, often, essential) fraction of discontent agents who are unable to find a better location. Models of the second subclass converge to simple stochastic equilibrium patterns that can be segregated or integrated depending on parameters. There may be other classifications of the Schelling-like models into those, which solutions stall and those, which solutions are simpler. However, the basic problem of the reviewed paper is that the Galvin’s et al (2009) version of the Schelling model belongs to the first class – its solutions are strongly path-dependent and, indeed, can stall in a great variety of unstable steady states. That is, in mathematical terms, the number of attractors of the version of Schelling model that is investigated in the paper is very large and the attractors’ patterns are complicated. One can follow here Chris Langton’s ideas of 1990s or recall Game of Life or elementary CA that served to Steven Wolfram for his “New Kind of Science” https://www.wolframscience.com/. Whatever, we cannot just average outputs of the dynamic systems for arbitrarily/randomly chosen set of parameters. Just as we cannot study the dynamics of the elementary CA until we recognize the correspondence between the rules that govern them and the patterns they produce. Compare the 50, 75, 86, etc. rules and completely different dynamic systems they generate http://mathworld.wolfram.com/ElementaryCellularAutomaton.html. At this point, the major idea of the paper is destroyed by: (1) path-dependent version of the Schelling (and Sugarscape) model; (2) extremely wide spectrum of initial conditions and (3) aggregate view of solutions. The authors overlook the consequences of the (1) – (3) and this takes them too far away from the correct view of the complex systems. The result is totally wrong result section. The observations of this section all describe averages over extremely heterogeneous sets of solutions that, just as the average characteristics of all elementary CA, they do not represent any meaningful phenomena.
%Minor comment: I have mentioned above some specific paper’s flaws. There are more.


%Referee: 2

%Comments to the Author

%I totally agree with the authors that “space matters” in models and this is one issue that makes geographical models complicated compared to say economic models that don’t consider spatial relationships in them. To me this is one of the first papers that to show systemically how initial spatial conditions (and variations in such) impact model results. By taking two well know agent-based models (Schelling’s Segregation model and Sugarscape) the authors are able to show quantitatively and qualitatively how different spatial conditions impact the results.

%However, I find it hard to follow your argument especially based on what you write: “We think simulation can become a very good tool to achieve this, provided that models do include relevant spatial descriptions and behavioural rules which take space into account, and provided that the model evaluation includes a sensitivity analysis of output variations to the way space is modelled.” Is unclear, in the sense, for example for your first condition, if I have a geographically explicit model of a real world place using GIS, say within an agent-based model, is space not directly accounted for and the behavior of the agents are also explicitly spelled out? There are many examples of this, as you note / cite in the paper. With respect to the second condition “provided that the model evaluation includes a sensitivity analysis of output variations to the way space is modelled” you write this is seldom met. In what sense? Maybe this might be the case for abstract models like the ones you have used here but how can you alter space in geographically explicit models using real data? Most geographically explicit models are focused on a specific area therefore how could one alter the spatial environments without applying it to another area which is often a difficult task and seldom done.

%Moreover, it is unclear in the current version of the paper if you turned off the “toroidal default setting” in NetLogo when using your different spaces in the Schelling model. I write this because if you are using realistic spaces does it make sense to have agents say at the south of the city also communicating with agents (i.e. checking their neighbors) in the north of the city? Which would be the case the Schelling NetLogo model, which would not be the case in reality.

% One of your key words you use “model validation” but in what sense?  You don’t really come back to this in the paper, also you write about “the validity of generative models is uncertain until their results are proven robust and representative of ’real-world’ conditions.” What do you consider real world conditions to be? You seem to allude that you need to have the spatial generator to generate realistic (styles) city shapes for testing on the Schelling model but I am not convinced.

%This seems to be superfluous on your main purpose of paper. Your paper could be much shorter if you just show how different spatial configurations impact the results rather than spending time talking about spatial patterns. Don’t get me wrong, I like how your spatial generator produces patterns seen within European cities and captures compact, polycentric, etc. urban forms under different parameter configurations but this appears to be moving away from the focus of the paper (at least as its portrayed in the introduction) specifically how initial spatial conditions impact the results.

%I guess, I am wondering if the paper could not be shorter and simpler if this generator was removed?  Or I would suggest you refocus the paper/ or better articulate in your introduction the objective, how different urban forms (initial conditions impact model results). You have clear objectives: “1/ to test the robustness of simulation results to small variations of meta-parameters and 2/ to study the non-trivial effects of typical categories of spatial distribution (monocentric vs. polycentric for example) on the results of a given model.” But your introduction does not lead clearly to this.

%Basically, I think the paper needs a more clearer focus, a better introduction which focus specifically on its objectives and clearer structure from background to results.






%Response to reviewers (Juste) :
%rev 1 :
%This review is quite arrogant ad totally misses the point of the paper, trying to fit a inappropriate framework on our work.
%- we do not study at all the attractors of the dynamic, but aggregated indicators to have a synthetic view of model behavior. The first remark is difficult to understand, and yes the spatial makes a difference (mostly in practice).
%- of course we deviate from dynamic system theory because we are not AT ALL in this theoretical framework. It is strange to try to fit a given framework to something that is positioned in a totally different one.
%- choice of the model : precisely it has not been shown systematically how model behavior depends on space, thus the choice. And these models fundamentally deal with space as it is in the core of agents dynamics.
%- the term phase diagram is used following the Gauvin paper, and we assume this extension of the concept.
%- the theoretical example of the saddle hypersurface makes no sense here in practice, as indicators we took precisely converge to narrow unimodal distributions (easily distinguishable when parameters vary)
%- we know that the attractor patterns are complicated, that's exactly why we use simulation and study aggregated indicators.
%- references on CA are not useful here (none of these models are CA and it would be difficult to fit within a CA fwk, by forcing the neighborhood to be all space). (looks like an argument from authority)
%- yes we can average macro outputs, that the standard way to do in simulation and empirically it shows no problem here -> maybe add a paragraph to show the intra-run variablity compared to variation between parameters, and convergence rate etc.
%- the conclusion are not "destroyed" as : (1) this is true for the model state, but we study aggregated outputs that are NOT path-dependent - did the reviewer entirely read the paper ? ; (2) idem, we average on initial distributions ; (3) idem. There is no "correct view of complex systems", there are different approaches, and yes we are far from dynamical systems because we take an other approach.
%- "do not represent meaningful phenomena" : seriously this reviewer never saw any work in simulation ; finding appropriate macro indicators for a model, that have a thematic sense (segregation in our case) is at the core of the study of simulation models.
%- please give us the other flaws that's the aim of a review (and 6 month to do a 1 page out-of-scope review, we could have expected more).

%rev 2 :
%- on the alteration of a real space, that is out of the scope of the current paper, but yes we aim at producing synthetic geographical configurations that capture a typical spatial configuration -> maybe explicit that in the discussion
%- on model validation -> yes we should precise the concept
% - we need to be more precise on the schelling implementation/specification
% - ok, we can be shorter on the generator (or put more explicitly that we provide it and the grids for other modelers to do similar analysis



%Response to reviewers (Marion) :

%rev1

%May“spatial” in “spatial conditions” make a difference?
%=> mouai, je ne suis pas non plus convaincue de la pertinence de cette remarque, à laquelle on répond partiellement dans la literature review. Idem que Juste, je pense qu’il s’agit d’un problème de cadrage théorique. En lien avec la dernière remarque du reviewer 2, je pense que l’on peut ajouter un court paragraphe sur le positionnement du papier en intro.

% cities’ “non-ergodicity” isa kind of a philosophical statements, some urban processes, as buildup area dynamics, can be indeed considered as path-dependent, while some, as urban traffic, are most probably, not.
% =>  On peut reprendre cette remarque pour illustrer la question de la non-ergodicité des villes ?

%The objectives of the paper demand very careful recognition between the system with path-dependent and path-independent dynamics.
% => Bon, ici comme on dit clairement que l’on est pas là pour discuter des comportements particuliers de ces deux modèles, je ne reprendrais pas les remarques ci-dessus. On peut les garder en mémoire si on nous repose la question lors de la 2 nd soumission du papier.


%  Sorry, phase diagram is a standard notion and it is something else, see https://en.wikipedia.org/wiki/Phase_diagram.
% => je préfère aussi citer Gauvin à Wikipedia ;)

% the reader can guess now that the intention of the author is to investigate the dynamics of the aggregate characteristics of the output as dependent on the parameter(s) P of the system. Putting aside wrong use of “phase diagram,” this is manageable, while quite standard.
% => « the reader can guess now that » : on doit le dire plus clairement et en amont

% That is, one cannot use median before system’s properties are known well enough.
% => Sa remarque n’est pas non plus sans justifications… je pense que l’on doit afficher plus clairement notre cadre d’analyse en introduction, pour que l’on ne nous reproche pas par la suite de ne pas avoir exploré entièrement le comportement du modèle avant de tirer des conclusions de l’analyse d’indicateurs agrégés. Mais on montre justement que certains espaces de paramètres sont plus ou moins sensibles aux variations de conditions spatiales initiales… je ne pense pas que la résolution analytique du modèle viendrait contredire une telle conclusion ?


% CAs and number of attractors 
% => Ok, on peut garder ce type d’éléments pour la partie discussion, en disant bien que le papier est un premier proof of concept et que notre but n’est pas d’étudier le comportement de ces deux modèles… je ne pense pas que l’on sur-interprétait nos résultats, mais seulement que l’on pointait les variations de sensibilité aux conditions initiales. Il me semble que le fait que cela soit dû à différents facteurs de variabilité du comportement du modèle ne pose pas trop de problèmes vu le niveau de généralité de l’analyse que l’on propose ?


% The observations of this section all describe averages over extremely heterogeneous sets of solutions that, just as the average characteristics of all elementary CA, they do not represent any meaningful phenomena.
% => idem que plus haut, avec en plus l’idée que on dit bien que la path-denpendency est justement une des raisons pour lesquelles la prise en compte de l’ensemble des conditions initiales dont l’espace est primordiale…


% rev 2 :

%Most geographically explicit models are focused on a specific area therefore how could one alter the spatial environments without applying it to another area which is often a difficult task and seldom done.
% => Justement, un papier comme celui de Isabelle Thomas et al. montre que même avec des modèles spatially explicit comme il dit, on doit prendre en compte l’effet de la configuration spatiale choisie (résolution de l’information géographique, effets de bord, etc.), bref c’est pas parce que tu as des données spatialisées que tu évites le MAUP, loin de là. En plus, ces modèles dressent des conclusions qui sortent souvent du cadre de leur espace d’études, sur la relation entre deux phénomènes par ex.

% Which would be the case the Schelling NetLogo model, which would not be the case in reality.
% => Euh, on utilise un torus nous ???


% model validation 
% C’est marrant, je dis souvent à mes étudiants que lorsqu’on utilise des « » ça met l’accent sur une notion non définie, on s’est fait piéger à notre tour ! La question est posée de savoir si l’on parle d’exploration du modèle ou de validation dans le papier ? La première solution nous éviterait de tomber dans un débat épistémologique sur les conditions nécessaires à la validation d’un modèle, et l’existence même d’un tel état de validation…

% spatial generator and focus
% Je suis d’accord avec cette partie, après il me semble que le générateur de formes urbaines est un vrai plus du papier et donc je le garderais à moins que d’autres reviewers nous fassent le même commentaire.


